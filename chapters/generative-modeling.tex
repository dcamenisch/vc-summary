\section{Generative Modeling}

In the previous part we looked at \textbf{discriminative models} with the aim to estimate the conditional distribution $p(y \; | \; x)$. Generative models aim to estimate the joint distribution $p(x, y)$. This will help us to model much more complex situations. Remember Bayes' rules:
$$p(y \; | \; x) = \frac{1}{z} \underbrace{p(y) \cdot p(x \; | \; y)}_{p(x,y)}$$

Where $z$ is the normalization constant $p(x)$. Generative modeling can be seen as the seen as the attempt to infer the process, according to which examples are generated.

\subsection{Naive Bayes Model}

We want to apply generative modeling for classification tasks. We starte by making the assumption that given some class label, each feature is independent of all the other features (therefore naive). This helps us estimating $p(\bar x \; | \; \bar y)$ as it is equal to $\prod_{i=1}^d p(x_i \; | \; y_i)$. \medskip

\subsection{Gaussian Naive Bayes Classifier}

We model the features by conditionally independent Gaussians and estimate the parameters via maximum likelihood estimation:
\begin{enumerate}
	\item MLE for class prior:
		$$p(y) = \hat p_y = \frac{\text{Count}(Y = y)}{n}$$
	\item MLE for feature distribution:
		$$p(x_i \; | \; y) = \mathcal{N}(x_i; \hat \mu_{y,i}), \sigma^2_{y,i}$$
\end{enumerate}

Where:
\begin{align*}
	\mu_{y,i} &= \frac{1}{\text{Count}(Y = y)} \sum_{j \; | \; y_j = y} x_{j,i}\\
	\sigma^2_{y,i} &= \frac{1}{\text{Count}(Y = y)} \sum_{j \; | \; y_j = y} (x_{j,i} - \hat \mu_{y, i})^2
\end{align*}

Predictions are then made by:
$$y = \argmax{\hat y} \; p(\hat y \; | \; x) = \argmax{\hat y} \; p(\hat y) \cdot \prod_{i=1}^d p(x_i \; | \; \hat y)$$

\begin{center}
	\includegraphics[width=0.8\columnwidth]{gnb.jpeg}
\end{center}

This is equivalent to the following decision rule for binary classification:
$$y = \sgn \underbrace{\left( \log \frac{p(Y = +1 \; | \; x)}{p(Y = -1 \; | \; x)} \right)}_{f(x)}$$

Where $f(x)$ is called the discriminant function. We can rewrite this and get:
\begin{align*}
f(x) &= \sum_{i=1}^d \underbrace{\frac{1}{\sigma_i^2} (\mu_{+1,i} - \mu_{-1,i})}_{w_i} \cdot x_i \\
&+ \underbrace{\log \frac{p}{1-p} + \sum_{i=1}^d \frac{1}{2 \sigma_i^2} (\mu_{-1, i}^2 - \mu_{+1, i}^2)}_{w_0}
\end{align*}

If the conditional independence assumption is violated, we can run into some serious issues, e.g. the classifier can become overconfident.

\subsection{Gaussian Bayes Classifier}

We drop the independence assumption and model our features as generated by a multivariant Gaussian $\mathcal{N}(x; \mu_y, \Sigma_y)$ with:
\begin{align*}
	\mu_{y} &= \frac{1}{\text{Count}(Y = y)} \sum_{j \; | \; y_j = y} x_{j}\\
	\Sigma_{y} &= \frac{1}{\text{Count}(Y = y)} \sum_{j \; | \; y_j = y} (x_{j} - \hat \mu_{y}) (x_{j} - \hat \mu_{y})^\top
\end{align*}

This is also called the \textbf{quadratic discriminant analysis} (QDA). 

\begin{center}
	\includegraphics[width=\columnwidth]{qda.jpeg}
\end{center}

If we impose the restriction that $\Sigma_+ = \Sigma_-$ this leads us to the linear discriminant analysis LDA and if we further restrict $p(y) = \frac{1}{2}$ we get the Fisher LDA.

\begin{center}
	\includegraphics[width=\columnwidth]{qda-bigpicture.jpeg}
\end{center}

Gaussian Bayes classifiers can also be used for outlier detection by introducing a threshold $\tau$ such that all data points $x$ with $p(x) \leq \tau$ are outliers.

\subsection{Avoiding Overfitting}

From previous examples we know that MLE is prone to overfitting. We can avoid this by employing the techniques already seen:
\begin{itemize}
	\item Restricting Model Class: fewer parameters (e.g. GNB)
	\item Using Priors: restrict ("smaller") parameter values
\end{itemize}

Using a prior for the parameters leads us again to MAP estimation.

\subsection{Generative vs. Discriminative}

Discriminative models:
\begin{itemize}
	\item Model $p(y \; | \; y)$ and do not attempt to model $p(x)$
	\item Cannot detect outliers
	\item Are typically more robust, since accurately modeling $x$ may be difficult
\end{itemize}

Generative models:
\begin{itemize}
	\item Model joint distribution $p(x,y)$ and are therefore more ambitious
	\item Can be more powerful (e.g. dectect outliers, missing values) if model assumptions are met
	\item Are typically less robust against outliers
\end{itemize}
